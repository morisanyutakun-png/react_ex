\documentclass[a4paper,11pt]{article}
\usepackage[margin=18mm]{geometry}
\usepackage{amsmath,amssymb}
\usepackage{enumitem}
\usepackage{fontspec}
\usepackage{xeCJK}

\IfFontExistsTF{Noto Sans CJK JP}{
\setCJKmainfont{Noto Sans CJK JP}
\setCJKsansfont{Noto Sans CJK JP}
\setCJKmonofont{Noto Sans Mono CJK JP}
}{
\IfFontExistsTF{Hiragino Sans}{
\setCJKmainfont{Hiragino Sans}
\setCJKsansfont{Hiragino Sans}
\setCJKmonofont{Hiragino Sans}
}{
\setCJKmainfont{IPAexGothic}
\setCJKsansfont{IPAexGothic}
\setCJKmonofont{IPAexGothic}
}
}

\setlength{\parindent}{0pt}
\setlist[enumerate]{itemsep=10pt}

\begin{document}

\section*{問題}
\begin{enumerate}
\item 二次関数 $f(x)=x^2-8x+7$ の最小値と,そのときの $x$ の値を求めよ。
\end{enumerate}

\clearpage

\section*{解答・解説}
\begin{enumerate}
\item \textbf{解答}: 最小値は $-9$,そのとき $x=4$。\
\textbf{解説}: 平方完成を用いる。
[
\begin{aligned}
f(x)
&=x^2-8x+7 \
&=\left(x^2-8x+16\right)+7-16 \
&=(x-4)^2-9
\end{aligned}
]
ここで $(x-4)^2\ge 0$ より,$(x-4)^2$ の最小値は $0$ である。したがって
[
f(x)=(x-4)^2-9 \ge 0-9=-9
]
よって最小値は $-9$。等号は $(x-4)^2=0$,すなわち $x=4$ のとき成り立つ。\
平方完成では,$x^2-8x$ に対して $+16$ を足した分を,必ず $-16$ として調整し,式全体が元と等しくなるようにする。
\end{enumerate}

\end{document}
