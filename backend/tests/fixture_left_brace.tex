\documentclass[a4paper,11pt]{article}
\usepackage[margin=18mm]{geometry}
\usepackage{amsmath,amssymb}
\usepackage{enumitem}
\usepackage{fontspec}
\usepackage{xeCJK}
\IfFontExistsTF{Hiragino Sans}{%
\setCJKmainfont{Hiragino Sans}
\setCJKsansfont{Hiragino Sans}
\setCJKmonofont{Hiragino Sans}
}{\IfFontExistsTF{Noto Sans CJK JP}{%
\setCJKmainfont{Noto Sans CJK JP}
\setCJKsansfont{Noto Sans CJK JP}
\setCJKmonofont{Noto Sans CJK JP}
}{}}
\setlength{\parindent}{0pt}
\setlist[enumerate]{itemsep=10pt}

\begin{document}

\section*{問題}
\begin{enumerate}
\item 二次関数 $f(x)=x^2-4x+3$ の最小値と,そのときの $x$ の値を求めよ。
\item 二次関数 $f(x)=x^2-6x+10$ の最小値と,そのときの $x$ の値を求めよ。
\item 二次関数 $f(x)=x^2+2x-5$ の最小値と,そのときの $x$ の値を求めよ。
\item 二次関数 $f(x)=2x^2-8x+1$ の最小値と,そのときの $x$ の値を求めよ。
\item 二次関数 $f(x)=-(x^2-4x+1)$ の最大値と,そのときの $x$ の値を求めよ。
\end{enumerate}

\clearpage

\section*{解答・解説}
\begin{enumerate}
\item \textbf{解答}: 最小値は $-1$,そのときの $x$ は $2$。\
\textbf{解説}: 平方完成を行う。
[
\begin{aligned}
f(x) &= x^2-4x+3 \
&= \left(x^2-4x+4\right)+3-4 \
&= (x-2)^2-1
\end{aligned}
]
ここで $(x-2)^2\ge 0$ より $f(x)\ge -1$。等号は $x=2$ のとき成り立つ。\
平方完成で $+4$ を足した分は,必ず $-4$ をして調整し,元の式と等しくなるようにする。

\item \textbf{解答}: 最小値は $1$,そのときの $x$ は $3$。\
\textbf{解説}: 平方完成を行う。
[
\begin{aligned}
f(x) &= x^2-6x+10 \
&= \left(x^2-6x+9\right)+10-9 \
&= (x-3)^2+1
\end{aligned}
]
ここで $(x-3)^2\ge 0$ より $f(x)\ge 1$。等号は $x=3$ のとき成り立つ。\
$+9$ を足したなら $-9$ で調整して,式全体が元と等しいことを保つ。

\item \textbf{解答}: 最小値は $-6$,そのときの $x$ は $-1$。\
\textbf{解説}: 平方完成を行う。
[
\begin{aligned}
f(x) &= x^2+2x-5 \
&= \left(x^2+2x+1\right)-5-1 \
&= (x+1)^2-6
\end{aligned}
]
ここで $(x+1)^2\ge 0$ より $f(x)\ge -6$。等号は $x=-1$ のとき成り立つ。\
平方完成で $+1$ を加えた分は,必ず $-1$ をして調整する。

\item \textbf{解答}: 最小値は $-7$,そのときの $x$ は $2$。\
\textbf{解説}: 係数をくくって平方完成を行う。
[
\begin{aligned}
f(x) &= 2x^2-8x+1 \
&= 2\left(x^2-4x\right)+1 \
&= 2\left{\left(x^2-4x+4\right)-4\right}+1 \
&= 2(x-2)^2-8+1 \
&= 2(x-2)^2-7
\end{aligned}
]
ここで $2(x-2)^2\ge 0$ より $f(x)\ge -7$。等号は $x=2$ のとき成り立つ。\
まず $2$ をくくり,その中で $+4$ を足した分を $-4$ で調整することに注意する。

\item \textbf{解答}: 最大値は $3$,そのときの $x$ は $2$。\
\textbf{解説}: まず中身を平方完成してから符号に注意して整理する。
[
\begin{aligned}
f(x) &= -(x^2-4x+1) \
&= -\left{\left(x^2-4x+4\right)+1-4\right} \
&= -\left{(x-2)^2-3\right} \
&= -(x-2)^2+3
\end{aligned}
]
ここで $(x-2)^2\ge 0$ より $-(x-2)^2\le 0$。したがって $f(x)\le 3$。\
等号は $(x-2)^2=0$,すなわち $x=2$ のとき成り立つ。\
$-(\ \ )$ が付くと大小関係が逆向きになる点に注意する。
\end{enumerate}

\end{document}
